% !TeX root = ../main.tex

% 中英文摘要和关键字

\begin{abstract}

    对程序复杂度和执行上界的分析由来已久。随着验证和综合技术的发展,自动化上界生成成为受到关注的研究方向。但是,现有工具所生成的上界的正确性并没有得到验证,实际中更是出现了生成上界为错误的情况。如何验证上界表达式的正确性是本文研究的主要问题。
    
    安全属性是形式验证特别关注的一类属性规范,描述了系统不会产生规定之外的行为。安全属性的检查已有长足的发展,近年在软件模型检测领域得到了广泛的应用。本文进行上界验证的基本思路是通过程序变换,将问题归约为安全属性的检查,从而利用现有的验证技术完成正确性的证明。
    
    本文在\texttt{Ultimate}框架内实现了验证算法。实验分析表明,本文方法能有效地验证程序的上界。

  % 关键词用“英文逗号”分隔,输出时会自动处理为正确的分隔符
  \thusetup{
    keywords = {上界验证, 安全属性检查, 形式化验证, 程序分析, 自动化验证},
  }
\end{abstract}

\begin{abstract*}
  Analyzing the complexity and execution bound of program has a long tradition. With the development of verification and synthesis technologies, automated bound analysis has become an interesting research field. However, the bound generated by existing analysis tools is not validated, which entails the possibility of erroneous bound. How to verify a bound expression is the main issue addressed in this paper.
  
  Of special interests in formal verification are safety properties, which assert the system never engenders any prohibited behavior. Safety properties checking has been massively studied and in recent years it has been used in software model checking. The methodology in the paper is to reduce bound verification to safety property checking via program instrumentation and utilize existing verification technologies.
  
  We implemented the bound verification algorithm in the \texttt{Ultimate} framework. Experimental analysis shows that our method can effectively verify program bound.

  % Use comma as seperator when inputting
  \thusetup{
    keywords* = {bound verification, safety property check, formal verification, program analysis, automated verification},
  }
\end{abstract*}
