% !TeX root = ../main.tex

\begin{acknowledgements}
  衷心感谢我的导师王生原副教授,以及软件学院贺飞副教授对本篇论文进行的悉心指导。没有两位老师的帮助,就没有这篇论文。
  
  形式化验证是计算机科学中历久弥新的研究方向,从计算机诞生之日起,人们就在寻找增强程序可信性的方法,时至今日已经发展出了各式各样的验证技术。它涉及诸多理论,又需要对实际程序和语言的理解,因此是一个理论和实践相结合的领域。在清华大学计算机系学习的四年时间里,我逐渐感受到形式化系统以及严谨证明之美,计算机于我不再是一个单纯的机器,而是一种计算模型。感谢计算机系诸位老师的言传身教,在这里我很幸运地确立了自己的兴趣,做出了自己的选择。
  
  2019年在王生原老师处进行SRT项目时,我第一次完整地接触了形式化验证,并系统学习了形式语义学和定理证明器的有关知识;王老师教授的自动机课程,也让我感受到计算机科学完全不同的一面。虽然很遗憾后续没有再继续进行定理证明相关的研究,但在此我想特别感谢王老师将我引入这一领域,您的启蒙对我起到非常重要的作用。
  
  我还想感谢计72班的谢兴宇同学。我们都对形式化验证感兴趣,在大三的两门操作系统课上一起完成了和OS验证相关的挑战性项目。他帮助我确认了后续的研究方向,我从他那里也学习到了很多知识。此外,疫情期间我在清华大学逻辑学中心学习,和那里的老师、同学度过了非常愉快的一学期,激发了我对形式逻辑的兴趣;在中科院软件所的实习经历也让我更加深入地了解了验证领域的前沿课题,在此一并表示感谢。
  
  最后感谢一直默默陪伴我的家人和朋友们,你们总是支持着我,构成了我坚强的后盾;感谢我挚爱的恋人,你照亮了我本科最后的岁月。
\end{acknowledgements}
