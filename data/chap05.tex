% !TeX root = ../main.tex

\chapter{结论}

本论文提出了程序中上界的形式化表示,以及将上界验证归约为安全属性检查问题的方法。这一自动上界验证的方法在确认算法复杂度、预测和改善程序性能以及终止性证明等方面都有一定的价值,尤其是在确证自动上界生成工具的可靠性方面很有必要。算法在\texttt{Ultimate}框架内得到了实现,并在真实程序集中进行了测试,结果显示这一方法能高效地证明或证伪大部分程序的上界。

不过本项工作的不足之处也是较为明显的。比如,从实际应用的角度来说,我们的工具由于是通过问题的归约进行验证,而安全属性检查又无法保证验证的成功,因此我们的工具也就没有相对完备性\footnote{问题本身是不可判定的,所以必然不完备。所谓相对完备(relative completeness)指的是在输入程序属于特定子类——如线性程序、确定程序等——的情况下保证求解。}。在很多验证问题中,将原问题归约为安全属性检查进行求解是比较直观的方法,但安全属性检查本身无法保证求解的问题也会随之影响上层的应用场景。究其原因,大抵在于安全属性本身表达能力较强,从而相应地判定性较弱。

因此,本论文进一步的工作方向之一便是采用其他验证方法来解决上界验证的问题。比如,在良好的约束编码情况下,基于SMT技术\cite{de_moura_satisfiability_2011}的约束求解方法可以保证较好的相对完备性,从而能拓宽工具的实际可用性;使用定理证明的方法能更准确地进行证明,但会牺牲效率等。

另一个可能的方向是放宽对程序的限制,支持更多类型的程序。比如,允许程序中存在函数调用,并在函数发生递归的情况下,通过跟踪递归的深度验证递归实例数量的上界,乃至于将函数间和函数内的验证结合起来,实现对大型实际程序和项目的支持。

上界的验证是一个一阶问题,而自动上界生成类似于程序综合(program synthesis\cite{gulwani_program_2017}),属于二阶问题,因此相对来说会更困难。我们可以和安全属性检查做一个对照:虽然它是一个一阶的验证问题,但对安全属性检查结果的验证,即对程序被证伪时给出的反例——以及与之相对应的,程序被验证为正确时所给出的证据(proof)——的验证,比安全属性检查本身要简单得多。但是对它们的验证是不可或缺的,人们逐渐认识到反例和证据的真实与否对建立工具的可靠性的意义。

验证(verification)与确证(validation)两者是相互作用的,自动上界生成也是如此。我们需要上界验证方法作为辅助,以保证工具的可靠性,甄别并杜绝计算出错误上界的情况。本论文即是在此意义上完成的。