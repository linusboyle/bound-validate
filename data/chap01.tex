% !TeX root = ../main.tex

\chapter{引言}

\section{研究背景}

自从通用计算模型\cite{turing_computable_1937}被提出以来,随着计算机科学和信息技术的发展,计算机程序已在各领域被广泛应用,逐渐成为社会生活中不可或缺的组成部分。人们通过形式语言来设计算法,描述计算机要执行的工作,但是模型缺陷或编码错误往往会引入造成严重后果的错误。研究程序正确性的验证方法因此具有重要意义。

\subsection{资源分析与上界分析}
\label{section:chap1-bound-analysis-intro}
程序的正确性是由一系列预期属性或规范所规定的。程序分析技术的目标即是通过对程序进行综合考察,分析程序的各类动态属性。资源分析(cost/resource analysis)是程序分析的一个研究方向,它假设程序有着一定的资源消耗模型(consumption model),期望通过对程序结构和语义的分析,预先得出或近似程序资源消耗量。

资源分析的一个主要应用场景是在嵌入式系统的分析中,得出实时控制系统的最坏情况执行时间\cite{wilhelm_worst-case_2008}(worst case execution time,简称WCET),此时,程序的执行时间被视为其消耗的“资源”;另外一个应用场景是分析安全攸关软件的信息安全级别,此时,系统中机密信息的泄漏程度被作为程序消耗的“资源”进行分析\cite{smith_foundations_2009}。

上界分析(bound analysis\footnote{直译为界分析,但实际场景中一般只会关注上界的存在})类似于对程序时间复杂度所做的资源分析,不过,上界分析一般采取更为抽象和高层次的复杂度表示。程序的执行时间,如上述的WCET指标,受到硬件实现细节的诸多影响,不利于理论的分析。现有的上界分析工具能从不同的粒度对上界进行分析,比如,从程序底层的变迁系统表示出发分析状态转移的次数,或者从程序控制流层面出发分析循环的迭代次数\cite{sinn_simple_2014},以及从函数间调用的层面分析递归的实例数目\cite{gulwani_speed_2009, gulwani_control-flow_2009}等。

\subsection{研究动机}

自动上界生成工具是以完全自动化的方式寻找输入程序的界。它们采取了不同的策略和技术,包括抽象解释技术、软件模型检测技术及计算代数理论等。其中,由奥地利维也纳技术大学的学者开发的\texttt{Loopus} 工具 \cite{sinn_simple_2014,zuleger_bound_2011,sinn_difference_2015,sinn_complexity_2017}是近期较为出色的工具。我们考察了其于2014年在测试集上进行的测验\footnote{原始测试结果见\url{https://forsyte.at/static/people/sinn/loopus/CAV14/comparisonLoopusKoATPUBSRank.html}},结果发现其计算得到的上界表达式出现了错误。比如,对于代码\ref{listing:chap1-motivating-program}中的C语言程序,\texttt{Loopus}给出的循环上界是$b - 1 + b \times (b - 1)$。

\begin{listing}

\begin{minted}
[
frame=lines,
framesep=2mm,
baselinestretch=1,
fontsize=\footnotesize,
linenos
]{c}
void ex02(int a, int b, int c) {
  a = 0;
  while (b >= 1 + a) {
      c = 0;
      while (1) {
        if (a >= c) {
          c = c + 1;
        }
        else if (c >= a + 1) {
          a = a + 1;
          break;
        }
        else if (1) {
          c = 1;
          a = 1;
          b = 1;
          break;
        }
        else
          return;
      }
  }
  return;
}
\end{minted}

\caption{一个\texttt{Loopus}所给出上界为错误的C语言程序}
\label{listing:chap1-motivating-program}
\end{listing}

循环上界将在下文形式化地定义,简单来说,其类似于程序中循环的迭代次数。显然,$b - 1 + b \times (b - 1)$并不是上述程序的上界,只需考虑当$b=1$时的情况\footnote{a和c的初始值不影响循环的迭代次数}:此时循环共迭代3次(内层循环2次,外层循环1次),多于带入表达式中所计算出的0次。

自动验证工具由于设计复杂,实现的难度较高,有可能出现编码问题,影响工具的可靠性,而确保验证工具的可靠性是至关重要的。比如,上例表明了在上界分析领域进行结果验证的必要性。对于自动上界生成工具所给出的结果,即上界表达式,需要一种方法检查其是否确为程序的上界,这对于增强系统的可靠性和置信度很有价值。本文即研究如何对程序的上界进行验证。

目前,没有专门进行上界验证的工作。有关自动上界生成领域的进展,参看附录\ref{cha:survey}的文献综述报告。

\section{本论文的结构}

本文的剩余部分安排如下:

第二章是前置定义,包含和上界验证相关的若干概念的形式化定义,包括程序的语法和语义、控制流图表示、自然循环、循环上界和安全属性检查等四个部分,此外还简要介绍了安全属性检查的原理。为力求完整和精确,这一章的形式化较多。

第三章提出基于安全属性检查的上界验证算法。首先关注的是单一循环的验证,分为全局上界和局部上界两个部分,并分别给出了问题归约方法的细节以及这一归约的正确性证明;其次通过“证明结构”这一概念,支持对多循环程序的验证。

第四章是算法实现和实验评估。本章首先概述了实现所基于的Ultimate框架及所用到的两种验证专用语言,其次介绍实现的细节。最后,给出了实验测试集的选择和筛选方法,以及工具在其上的实验结果。

第五章是结论和展望,对于目前工作进行了总结,对进一步工作的可能方向进行了讨论。

